Propositional logic is reasonably straight forwards. You make a bunch of assertions and test if they are true or not. Let's check out the syntax whilst trying to keep it nice and simple.

\section{Syntax}
An element or assertion in logic (e.g., "it is sunny today", "I slept through my lecture") is called a \gls{proposition}. They're normally given single letters like \textit{p} or \textit{q} (p for proposition, right?), but as with all things maths, definitions change when the author feels like it.

If you have the proposition on its own, it's known as an \gls{atomicfmla}. However, you can connect more than one proposition together using a \gls{bincon}, which are a way of creating a new proposition from two propositions (there's also a unary connective, which is the not connective, or the negator ($\neg$)). Here, let's try this out, shamelessly \sout{stolen} borrowed from Hirsch's notes.

\begin{itemize}
    \item $p$
    \item $(p \lor q)$
    \item $\neg(p \land q)$
    \item $(\neg p \to (\neg q \lor r))$
\end{itemize}
